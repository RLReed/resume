%%%%%%%%%%%%%%%%%%%%%%%%%%%%%%%%%%%%%%%%%
% Medium Length Graduate Curriculum Vitae
% LaTeX Template
% Version 1.1 (9/12/12)
%
% This template has been downloaded from:
% http://www.LaTeXTemplates.com
%
% Original author:
% Rensselaer Polytechnic Institute (http://www.rpi.edu/dept/arc/training/latex/resumes/)
%
% Important note:
% This template requires the res.cls file to be in the same directory as the
% .tex file. The res.cls file provides the resume style used for structuring the
% document.
%
%%%%%%%%%%%%%%%%%%%%%%%%%%%%%%%%%%%%%%%%%
\vskip-20pt%
%----------------------------------------------------------------------------------------
%	PACKAGES AND OTHER DOCUMENT CONFIGURATIONS
%----------------------------------------------------------------------------------------

\documentclass[margin, 10pt]{res} % Use the res.cls style, the font size can be changed to 11pt or 12pt here

\usepackage{helvet} % Default font is the helvetica postscript font
%\usepackage{newcent} % To change the default font to the new century 
%schoolbook %postscript font uncomment this line and comment the one above

\setlength{\textwidth}{5.1in} % Text width of the document

\begin{document}

%----------------------------------------------------------------------------------------
%	NAME AND ADDRESS SECTION
%----------------------------------------------------------------------------------------

\moveleft.5\hoffset\centerline{\Large\bf Richard Reed Jr} % Your name at the top
 
\moveleft\hoffset\vbox{\hrule width\resumewidth height 1pt} \vskip-10pt%
\moveleft\hoffset\vbox{\hrule width\resumewidth height 
0.15pt}\smallskip 
%Horizontal line after name; adjust line thickness by changing the '1pt'
 
\moveleft.5\hoffset\centerline{1420 Watson Pl 14 \hfill (620) 
755-5202} % Your address
\moveleft.5\hoffset\centerline{Manhattan, KS 66502 \hfill rlreed@ksu.edu}

%----------------------------------------------------------------------------------------

\begin{resume}
 
\section{OBJECTIVE}  

Experience in the design and implementation of computer programs in the field 
of nuclear engineering with applications in numerical solvers.

\section{EDUCATION}

{\sl Doctorate of Philosophy,} Nuclear Engineering \hfill Expected Dec 2018\\
Kansas State University, Manhattan, KS \\
Cumulative GPA: 3.895/4.000

{\sl Graduate Certificate in Applied Mathematics,} Nuclear Engineering \hfill Dec 2016\\
Kansas State University, Manhattan, KS

{\sl Master of Science,} Nuclear Engineering \hfill May 2015\\
Kansas State University, Manhattan, KS \\
Cumulative GPA: 3.923/4.000

{\sl Bachelor of Science,} Chemical Engineering \hfill May 2011 \\
Kansas State University, Manhattan, KS \\
Completed University Honors Program \\
Secondary Major: Biological Engineering \\
Cumulative GPA: 3.541/4.000

\section{RELEVANT \\ EXPERIENCE}

{\sl Graduate Teaching Assistant} \hfill Fall 2014 - Present \\
Mechanical and Nuclear Engineering, Kansas State University, Manhattan, KS \\
\begin{itemize} \itemsep -2pt % Reduce space between items
\item Taught weekly recitation session for NE 495 (intro to NE)
\item Led the Lab sessions for undergraduate and graduate Python programming courses
\item Developed practice problems and exam questions in conjunction with 
professor
\end{itemize}
 
{\sl Graduate Research Assistant} \hfill Fall 2013 - Present \\
Mechanical and Nuclear Engineering, Kansas State University, Manhattan, KS
\begin{itemize} 
\item Research Topics include:
\begin{itemize}
	\item Improving the Discrete Generalized Multigroup method in Fortran by means of basis expansion and spatial homogenization
    \item Developing python script to write input files for the KSU Triga Mark II reactor in MCNP, Serpent, and Keno
\end{itemize}
\end{itemize}

{\sl Student Scientist} \hfill Summer 2016 \\
Computational Physics and Methods Group (CCS-2), LANL, Los Alamos, NM
\begin{itemize} 
\item Implemented the Discrete Diffusion Monte Carlo method applied to an arbitrary triangular mesh in Python
\end{itemize}

\section{COMPUTER \\ SKILLS} 

{\sl Languages \& Software:} 
Python, Fortran90, VBA, C++, Microsoft Office Suite, MCNP6, Serpent2, Scale6.1 (KENO), Git, \LaTeX  \\
{\sl Operating Systems:} Linux Mint, Windows 8, Windows 7, CentOS, Windows XP

\section{PROFESSIONAL \\ AFFILIATIONS  \\ \& ACTIVITIES} 

Student member of American Nuclear Society (ANS) \\
Member of Alpha Nu Sigma, Nuclear Engineering Honorary \\
Member of Omega Chi Epsilon, Chemical Engineering Honorary\\
Fellowship for the Computational Physics Summer Workshop 2015 at LANL\\
Awarded the Kansas State University Nuclear Research Fellowship

\section{PUBLISHED \\ WORKS}
Full Publications
\begin{itemize}
\item Reed, Richard L., and Jeremy A. Roberts. ``Effectiveness of the Discrete Generalized Multigroup Method based on truncated, POD-driven basis sets.'' Manuscript submitted for publication June 2018
\item Reed, Richard L., and Jeremy A. Roberts.``Application of the Karhunen--Lo\`eve Transform to the C5G7 benchmark in the response matrix method." Annals of Nuclear Energy 103 (2017): 350-355.
\item Richard L. Reed, and Jeremy A. Roberts. ``An energy basis for response matrix methods based on the Karhunen–Loéve transform." Annals of Nuclear Energy 78 (2015): 70-80.
\item Richard Reed. ``Applications of the Karhunen-Loéve Transform for Basis Generation in the Response Matrix Method". Diss. Kansas State University, 2015.
\item Jeremy A. Roberts, Richard L. Reed, and Benoit Forget. ``Phase space bases for response matrix methods". No. JAEA-CONF--2014-003. 2015.
\end{itemize}
Conference Transactions
\begin{itemize}
\item Richard L. Reed, Jeremy A. Roberts. (2017, Nov). ``Application of Truncated Karhunen-Loève Transform Basis Sets in the 1-D Discrete Generalized Multigroup Method", 2017 Winter Meeting of American Nuclear Society, Washington D.C.
\item Richard L. Reed, Jacob W. Hayhurst, Shravan D. Gangadhara, Jeremy A. Roberts. (2016, April). ``Updating a PWR Simulator in Python", 2016 Student Conference of the American Nuclear Society, Madison, WI.
\item Richard L. Reed, Jeremy A. Roberts. (2016, Nov). ``Application of the Karhunen-Lo\'{e}ve Transform to the C5G7 benchmark in the Response Matrix Method", 2015 Winter Meeting of American Nuclear Society, Washington D.C.
\item Richard L. Reed, Frank G. VanGessel, Mathew Cleveland, Allan Wollaber, Todd Urbatsch. (2015, Aug). ``Monte Carlo Thermal Radiation
Transport: Discrete Diffusion Monte Carlo (DDMC) on Triangular Mesh", 2015 Computational Physics Student Summer Workshop, Los Alamos, NM. Los Alamos National Laboratory
\item Richard L. Reed, Jeremy A. Roberts. (2014, June). ``Energy Expansion in Response Matrix Methods Using the Karhunen-Lo\'{e}ve Transform", 2014 Annual Meeting of American Nuclear Society, Reno, NV.
\end{itemize}
% Posters
% \begin{itemize}
% \item Richard L. Reed, and Jeremy A. Roberts (2016, Nov). ``Accuracy of the Discrete Generalized Multigroup Method using Truncated Basis Sets". Poster presented at the 2016 Winter Meeting of American Nuclear Society, Las Vegas, NV.
% \item Richard Reed, Jacob Hayhurst, Shravan Gangadhara, Brad Schoonover, Satya Sager Vanteddu, Jeremy Roberts (2016, June). ``Updating a PWR Simulator in Python". Poster presented at the Annual Meeting of American Nuclear Society, New Orleans, LA.
% \end{itemize}



\end{resume}
\end{document}